\section{Metodologia}

	Nessa primeira fase do projeto foi realizado um estudo sobre os conceitos básicos envolvendo workflows, workflows científicos, Sistemas  Gerenciadores de Workflows Científicos(SGWC), Redes de Petri, ferramentas de representação de redes de Petri, modelagem analítica de workflows científicos(como o Montage) e Simuladores de workflows científicos. 
	%achar uma melhor forma para dizer essas coisas.
	Depois do estudo da bibliografia básica envolvendo workflow científicas e as motivações acadêmicas para a pesquisa de workflows científicas, foi feito o estudo de alguns SGWC, quanto a implementação de workflows e a representação do fluxo de dados em cada um deles.
	
	Para aparar o formalismo matemático evolvendo workflows científicos e fazer uma modelagem analítica de alguns workflows bem conhecidos foram estudas Redes de Petri, Redes de Petri Temporizadas e Redes de Petri Estocásticas, assim como uma ferramenta para representação gráfica e análise das RdPEs (PIPE).
	
	Para finalizar a primeira parte do trabalho foi feito um estudo do WorkFlowSim, quanto possibilidade de ferramenta de simulação para o projeto.
			
	%Depois do estudo de workflows científicos e suas características, para se extrair índices de desempenho dos workflows científicos e ter possibilidade de ser criar no futuro um algoritmo capaz de modelar analiticamente um workflow simples no forma DAX precisamos conhecer a estrutura de comportamento dos fluxo de dados dos workflows e caracterizar esse fluxo de dados extrair em alguns padrões de comportamento % colocar as possibilidades de fluxo de dados 
	%Assim modelamos em Redes de Petri um workflow bem conhecido pela conhecido pela comunidade o montagem como pode ser visto da figura abaixo:

	%Observe que apareceram algumas estruturas para modelar em Redes de Petri o Montage, isso mostra o poder dessa teoria em modelagem analítica de workflows científica, como foi visto acima com redes de Petri é possível criar sincronização, paralelismo...[Continuar depois]
	%Isso 
	
%A partido dos estudos... é possível automatizar a conversão de de modelos em DAX, sem grandes problemas.
