\section{Conclusões Parciais}
		
	 Todas as tarefas previstas no cronograma projeto foram realizadas com sucesso e estão listadas abaixo, assim como as atividades em andamento e as atividades futuras:

\subsection{Tarefas Realizadas}

	\begin{itemize}
			\item Estudo dos conceitos básicos relacionados a workflows científicos e seus respectivos modelos de representação de fluxo de dados;
			\item Estudo dos conceitos básicos relacionados à modelagem estocástica de workflows científicos;

			\item Identificação e caracterização do conjunto de construtores de fluxo de dados: Kepler, Taverna e Pegasus.
			\item Modelagem do Montage utilizando o PIPE;
			\item Estudo básido dos possíveis simuladores de workflows científicos para validar o modelo analítico criado.
	\end{itemize}

\subsection{Tarefas em Andamento}
		\begin{itemize}
		
			\item \textbf{2 meses} - Estudo da viabilidade do uso de diferentes tipos de formalismos estocásticos(em particular Redes de Petri e as Álgebra de processos estocásticos) para a modelagem de fluxos de dados;
			
			\item \textbf{15 dias} - Estudo do funcionamento do simuladores de workflows: WorkflowSim e Cloud-DVFS, quanto possibilidade de representação um ambiente computacional que possa ser modelado analiticamente;
		\end{itemize}

\subsection{Tarefas a serem realizadas}
		\begin{itemize}
			\item \textbf{15 dias}-Estudo e definição de índices extraídos a partir dos modelos representados na etapa anterior, que possam refletir um impacto no desempenho do fluxo de dados;
			\item \textbf{15 dias}- Montar uma simulação de workflow científico em um dos simuladores estudado na etapa anteriores;
			\item \textbf{1,5 meses}-Desenvolvimento de um algoritmo que faça a conversão automática de fluxo de dados de um fluxo de dados de workflow em um modelo estocástico correspondente(selecionado dentre os tipos de modelos estudados).
		\end{itemize}
 