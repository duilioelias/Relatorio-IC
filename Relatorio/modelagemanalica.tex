%ataptar texto
	
	Ao se usar as redes de Petri para se modelar a execução de um workflow, cada
vértice pode armazenar um ou mais tokens. Diferentemente de sistemas mais tradicionais de processamento de dados, que comumente processam somente um único
fluxo de tokens entrantes, os vértices de transição de redes de Petri podem consumir e produzir tokens de múltiplas localizações. Os vértices de transições agem em tokens de entrada por um processo denominado disparo. Quando uma transição é disparada, ela consome os tokens de suas posições de entrada, realiza alguma tarefa de processamento,e realoca um número específico de tokens nas suas posições de saída. Isso é feito atomicamente. Como os disparos são não determinísticos, as redes de Petri são muito utilizadas para modelar comportamento concorrente em sistemas distribuídos.

