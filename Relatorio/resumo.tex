\begin{resumo}

%----
% Estrutura básica de um resumo
	%Contextualização breve
	%Objetivos
	%Métodos
	%Resultados
	%Conclusões
%----
	Nas últimas décadas os workflows científicos receberam grande atenção da comunidade acadêmica, devido à sua grande capacidade de manipulação, análise e simulação de experimentos que trabalham com grandes quantidades de dados. No entanto, as ferramentas existentes para tal fim são ainda escassas e muitos cientistas afirmam que isso é um grande gargalo da produção científica. Neste trabalho, busca-se contribuir com este processo, a partir da identificação e caracterização de linguagens de modelagem de workflows científicos, criando modelos estocásticos capazes de descrever analiticamente alguns workflows bem estudado pela comunidade acadêmica. Com isso, pretende-se extrair índices capazes de predizer o impacto do uso de diferentes abordagens de controle de fluxo de dados no desempenho dos workflows e compara-los com os dados das simulações.

 \vspace{\onelineskip}
 
 \noindent
 \textbf{Palavras-chaves}: Workflows científicos. Redes de Petri. Análise de Desempenho. Gerenciadores de Workflows.
\end{resumo}
