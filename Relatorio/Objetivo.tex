\section{Objetivo}

		Neste trabalho, pretende-se identificar e caracterizar os principais construtores de fluxo de dados encontrados em linguagens de modelagem de workflows científicos, bem como criar modelos estocásticos em Redes de Petri capazes de predizer o impacto causado pelo fluxo de dados no desempenho da execução de workflows. Para validar os índices que espera-se obter do modelo estocástico criado, serão modelados em Redes de Petri um conjunto de workflows científicos bem conhecidos pela comunidade científica, como o Montage, CyberShake, Epigenomics e SIPHT \footnote{https://confluence.pegasus.isi.edu/display/pegasus/WorkflowGenerator}. As predições obtidas a partir do modelo de Redes de Petri são comparadas com as obtidas a partir da simulacão da execucão dos mesmos workflows na ferramenta de simulacão WorkflowSim\footnote{http://www.workflowsim.org/}.	Com os dados obtidas nas fases anteriores pretende-se criar um algoritmo capaz de gerar modelos estocásticos a partir de modelos de fluxos de dados de workflows científicos.
