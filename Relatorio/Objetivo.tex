\section{Objetivo}

		O objetivo deste trabalho foi identificar e caracterizar os principais construtores de fluxo de dados encontrados em linguagens de modelagem de workflows científicos, bem como criar modelos estocásticos em Redes de Petri capazes de predizer o impacto causado pelo fluxo de dados no desempenho da execução de workflows. Os estudos conduzidos no trabalho se basearam em um conjunto de modelos de workflow científicos que são amplamente usados na avaliação de sistema gerenciadores de workflows. As predições obtidas a partir do modelo de Redes de Petri foram comparadas com as obtidas a partir da simulação da execução dos mesmos workflows na ferramenta de simulação WorkflowSim\footnote{http://www.workflowsim.org/}.
