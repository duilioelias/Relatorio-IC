	\subsection{Simuladores de workflows científicos}	
			
	Depois de criados os modelos em RdPEs para workflows científicos há algumas maneiras de validar o modelo, a primeira possibilidade é execução do workflows. Essa alternativa possui como problema o custo monetário envolvido na instanciação das máquinas, o que a torna inviável no primeiro momento. Uma outra alternativa mais viável é a  utilização de um ambiente de simulação computacional. Essa alternativa abre a possibilidade de criar-se diferentes ambientes computacionais e facilidade de reprodução da simulação, o que torna viável a busca por gargalos no workflows utilizados. Além da possibilidade avaliar modelos analíticos em um ambiente controlado que é um dos objetivos desse trabalho.
	
	Nesse trabalho utilizou-se o WorkflowSim\cite{chen:workflowsim} que é uma ferramenta de simulação de workflows científicos baseada no CloudSim\cite{google:cloudsim}, que também é uma ferramenta de simulação. No entanto, o CloudSim não possui certa características presentes no WorkflowSim, por exemplo, a simulação de ambientes heterogêneos e possibilidade de incluir falhas no sistema. As atividades no WorkflowSim  são descritas na linguagem DAX (linguagem para a descrição de workflows em formato XML usada pelo SGWC Pegasus). O WorkflowSim, possibilita também, modelar todo um ambiente computacional para execução dos workflows(Quantidade de \textit{Virtual Machine(VM)} e \textit{DataCenters}). Observe no Anexo ~\ref{Anexo2} como é descrito um ambiente computacional no WorkflowSim. 
							
				
