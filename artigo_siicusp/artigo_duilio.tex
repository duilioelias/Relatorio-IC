\documentclass[a4paper,10pt]{article}
\usepackage[left=2cm,right=2cm,top=3.0cm,bottom=3cm]{geometry}
\usepackage[brazilian]{babel}
\usepackage[utf8]{inputenc}
\usepackage[T1]{fontenc}
\usepackage{graphicx}
\usepackage{amssymb}
\usepackage{amsmath}
\usepackage[colorlinks=true,linkcolor=black,linktoc=all, urlcolor=black]{hyperref}
\usepackage{fancyhdr}

%\usepackage{helvet}
\renewcommand{\familydefault}{\sfdefault}


\date{}
\title{
    O Uso de Redes de Petri na Modelagem de \\Workflows Científicos e seus Recursos\footnote{Este trabalho foi financiado por uma bolsa de iniciação científica RUSP (processo número: ?????).}
}

\author{
\textbf{\textit{Duílio Henrique Haroldo Elias}},\textbf{ \textit{Kelly Rosa Braghetto}}\\
\\
\textit{Universidade de São Paulo} / \textit{Instituto de Matemática e Estatística}\\
\\
\href{mailto:duilio.elias@usp.br}{duilio.elias@usp.br} | \href{mailto:kellyrb@ime.usp.br}{kellyrb@ime.usp.br}
}


% Define o rodapé e o cabeçalho
\fancyhf{} 
\renewcommand{\headrulewidth}{0pt}
\cfoot{SIICUSP 2014 -- 22º Simpósio Internacional de Iniciação Científica e Tecnológica da USP }
\pagestyle{fancy}

\begin{document}
\maketitle
    
\section*{Resumo}
%*** Na primeira página deve ficar apenas os resumos em português e em inglês. \\
%O texto deve ter um total de 4 páginas. ***
	Nas últimas décadas os workflows científicos receberam grande atenção da comunidade acadêmica, devido à sua grande
capacidade de manipulação, análise e simulação de experimentos que trabalham com grandes quantidades de dados. No entanto,
as ferramentas existentes para tal fim são ainda escassas e muitos cientistas afirmam que esse seja o é um grande gargalo da produção científica. Neste trabalho, busca-se contribuir com este processo, criando modelos estocásticos capazes de descrever analiticamente alguns workflows bem estudado pela comunidade acadêmica. Com isso, pretende-se extrair índices capazes de predizer o impacto do uso de diferentes abordagens de controle de fluxo de dados no desempenho dos workflows e compara-los com os dados das simulações.

 
\section*{Abstract}

blablablab

blablablab

blablablab

\thispagestyle{fancy}

\newpage

\section*{Introdução}

Falar sobre:
\begin{itemize}
	\item O que são workflows científicos(OK) e porque eles são importantes(acrescentar algo)
	\item Como eles são modelados? Que ferramentas de apoio eles possuem
	\item Porque é interessante modelar workflows usando linguagens formais como Redes de Petri (verificação de propriedades dos modelos, extração de índices de desempenho, etc.)
	\item Porque a predição de desempenho é importante? Onde ela pode ser usada?
	\item Porque é difícil usar linguagens formais na modelagem? 
\end{itemize}
	
	Um workflow científico(WfC) pode ser compreendido como um conjunto de tarefas computacionais que constituem um experimento científico \cite{Junior2012}, ou seja, os passos a serem executados pelo experimento. Os WfCs são conhecidos pelo seu poder de manipulação e execução de experimentos científicos automatizáveis que trabalham com grandes quantidades de dados. As principais etapas do ciclo de vida de um WfC são a modelagem, execução e análise e são normalmente gerenciadas por sistemas complexos chamados de Sistemas Gerenciadores de Workflows Científicos(SGWfC).
	

\section*{Objetivos}

Detalhar um pouquinho as seguintes ideias:
\begin{itemize}
\item O objetivo do trabalhar é definir um método de conversão de modelos de workflows baseados em fluxos de dados em modelos em Rede de Petri Estocásticas
\item Em modelos em Rede de Petri Estocásticas, pode-se também incorporar informações sobre o tempo de execução das atividades e também sobre os recursos computacionais disponíveis para a execução do workflow; por essa razão, é possível extrair deles prediçoes sobre o desempenho do workflow.
\end{itemize}

Como a parte de comparar os resultados obtidos com as RdPs com os obtidos por meio do simulador não foi finalizada a tempo, não vamos mencioná-la aqui na seção de objetivos. Mas podemos falar dela na seção de trabalhos futuros.

\section*{Materiais e Métodos}

Aqui a ideia é explicar os conceitos importantes relacionados ao seu trabalho e descrever de forma sucinta o que você fez:

\begin{itemize}
\item Explicar sucintamente os componentes de uma RdP
\item Falar sobre as construções de fluxo de dados básicas (pipeline, distribuição, agregação)
\item Para cada uma das construções acima, mostrar como ela é mapeada na RdP; também explicar que, no modelo em RdP, uma atividade é modelada como uma transição
\item Dizer que, como estudo de caso, foi usado um conjunto de workflows científicos sintéticos (extraídos de workflows reais) que ilustram o uso de diferentes construções de fluxos de dados e que com frequência são empregados como base de comparaçã. Citar a fonte desses workflows.  
\item Dizer que os workflows estavam descritos na linguagem DAX; seu modelos já trazem informações sobre o tempo médio de execução de cada atividade e também sobre o tamanho dos dados de entrada. Essas informações são importantes na construção do modelo estocástico em RdP
\item Explicar como informações sobre os recursos computacionais disponíveis podem ser incluídas no modelo em RdP
\item Falar que foi usada a ferramenta PIPE para a análise numérica do modelo.
\item Dizer que a análise numérica do modelo nos dá de imediato dois índices de desempenho interessantes: a porcentagem de utilização dos recursos e o rendimento (throughput) das atividades e do workflow.    
\end{itemize}
\section*{Resultados}

Aqui não temos muita coisa para explicar.
Temos que tentar incluir um ou dois daqueles workflows que você fez em Redes de Petri e depois dizer que os demais modelos estão no site (e colocar o link para aquela página que eu criei: \url{http://www.ime.usp.br/~kellyrb/ic/#duilio}).

\section*{Conclusões}

Aqui é preciso dizer coisas como:
\begin{itemize}
\item No trabalho, foram modelados em RdP um conjunto de workflows científicos sintéticos (extraídos de workflows reais) que ilustram o uso de diferentes construções de fluxos de dados 
\item Concluiu-se que, por meio das Redes de Petri, é possível representar todas as construções básicas de fluxos de dados
\item Recursos computacionais simples (como máquinas, VMs, processadores) são facilmente expressos nos modelos em RdP
\item A solução numérica dos modelos em RdP traz uma predição do desempenho do workflow (utilização dos recursos e o rendimento (throughput) das atividades e do workflow)
\item Dizer que, como trabalho futuro, intenciona-se comparar os resultados obtidos a partir das redes de petri com resultados obtidos por meio da simulação do workflow em ferramentas como o WorkflowSim.
\end{itemize}

\bibliographystyle{bababbr3}
\bibliography{referencias}

\end{document}
